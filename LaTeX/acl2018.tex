%
% File acl2018.tex
%
%% Based on the style files for ACL-2017, with some changes, which were, in turn,
%% Based on the style files for ACL-2015, with some improvements
%%  taken from the NAACL-2016 style
%% Based on the style files for ACL-2014, which were, in turn,
%% based on ACL-2013, ACL-2012, ACL-2011, ACL-2010, ACL-IJCNLP-2009,
%% EACL-2009, IJCNLP-2008...
%% Based on the style files for EACL 2006 by 
%%e.agirre@ehu.es or Sergi.Balari@uab.es
%% and that of ACL 08 by Joakim Nivre and Noah Smith

\documentclass[11pt,a4paper]{article}
\usepackage[hyperref]{acl2018}
\usepackage{times}
\usepackage{latexsym}
\usepackage{natbib}

\usepackage{url}

%\aclfinalcopy % Uncomment this line for the final submission
%\def\aclpaperid{***} %  Enter the acl Paper ID here

%\setlength\titlebox{5cm}
% You can expand the titlebox if you need extra space
% to show all the authors. Please do not make the titlebox
% smaller than 5cm (the original size); we will check this
% in the camera-ready version and ask you to change it back.

% my_packages
\usepackage[utf8]{inputenc}
\usepackage{graphicx}
\usepackage{color}

\newcommand\BibTeX{B{\sc ib}\TeX}

\title{Deconvolution : a deep tool box for linguistic analysis}

\author{
L. Vanni\textsuperscript{1}, V. Elango\textsuperscript{2}, C. Aguilar\textsuperscript{1}, D. Longrée\textsuperscript{3}, D. Mayaffre\textsuperscript{1}, F. Precioso\textsuperscript{2}, M. Ducoffe\textsuperscript{2}\\
  \textsuperscript{1} Univ. Nice Sophia Antipolis - I3S, UMR UNS-CNRS 7271 06900 Sophia Antipolis, France \\
  \{lvanni, mayaffre\}@unice.fr \\
  \textsuperscript{2} Univ. Nice Sophia Antipolis - BCL, UMR UNS-CNRS 7320 - 06357 Nice CEDEX 4, France \\
  \{ducoffe, precioso\}@unice.fr - ecveer@gmail.com \\
  \textsuperscript{3} Univ. Liège - L.A.S.L.A, Bélgique \\
  dominique.longree@uliege.be\\}
\date{}

\begin{document}
\maketitle
\begin{abstract}
  This document contains the instructions for preparing a camera-ready
  manuscript for the proceedings of ACL 2018. The document itself
  conforms to its own specifications, and is therefore an example of
  what your manuscript should look like. These instructions should be
  used for both papers submitted for review and for final versions of
  accepted papers.  Authors are asked to conform to all the directions
  reported in this document.
\end{abstract}

\section{Introduction}
As in many other fields of data analysis, Natural Language Processing (NLP) has been strongly impacted by the recent advances in Machine Learning, more particularly with the emergence of Deep Learning techniques. These techniques outperform all other state-of-the-art approaches on a wide range of NLP tasks and so they have been quickly and intensively used in industrial systems. Such systems rely on end-to-end training on large amounts of data, making no prior assumptions about linguistic structure and focusing on stastically frequent patterns. Thus, they somehow step away from computational linguistics as they learn implicit linguistic information automatically without aiming at explaining or even exhibiting classic linguistic structures underlying the decision.

This is the question we raise in this article and that we intend to address by exhibiting classic linguistic patterns which are indeed exploited implictly in deep architectures to lead to higher performances.
Do neural networks make use of co-occurrences and other standard features, considered in traditional Textual Data Analysis (TDA)? 
Do they also rely on complementary linguistic structure which is invisible to traditional techniques? If so, projecting neural networks 
features back onto the input space would highlight new linguistic structures would lead to improving the analysis of a corpus and a better understanding on where the power of the Deep Learning techniques comes from.
Our hypothesis is that deep learning is sensitive to the linguistic units on which the computation of the key statistical sentences is based as well as to phenomena than other than frequency and complex linguistic 
observables. The TDA has more difficulty taking such elements into account -- such as linguistic linguistic patterns (Mellet et Longrée, 2009).
Our contribution confronts Textual Data Analysis and Convolutional Neural Networks for text analysis. 
We take advantage of deconvolution networks for image analysis in order to present a new perspective on text analysis to the linguistic community that we call deconvolution saliency. Our deconvolution saliency corresponds to the sum over the 
word embedding of the deconvolution projection of a given feature map. Such score provides a heat-map of 
words in a sentence that highlights the pattern relevant for the classification decision.
We examine z-scoring and deconvolution saliency on three languages: English, 
French and Latin. For all our datasets, deconvolution saliency highlights new linguistic observables, invisible with z-scoring alone.




\section{Related work}

Mauris ut magna ut diam hendrerit tincidunt. Duis turpis lacus, lacinia ut accumsan a, rutrum eu justo. Donec efficitur purus non leo iaculis elementum. Nulla pulvinar ligula ut pretium vulputate. Mauris non suscipit felis, ac molestie sem. Duis quis lacus sed massa pharetra eleifend non ut urna. Phasellus lobortis mattis pharetra. Phasellus mattis purus non quam molestie tincidunt.

Sed hendrerit at leo sed tristique. Vestibulum fringilla, nisi id rutrum congue, erat elit hendrerit mi, eget tempus erat nisl sit amet urna. Pellentesque ornare, nunc vel molestie scelerisque, ante augue condimentum orci, sollicitudin sodales odio neque at tortor. Mauris pellentesque ex neque, ut finibus ante pulvinar nec. Lorem ipsum dolor sit amet, consectetur adipiscing elit. Phasellus et fermentum urna. Donec rutrum, ex vel interdum mattis, nibh augue mattis magna, vel varius ante ipsum sed diam. Sed dui risus, gravida sit amet ornare nec, lacinia id elit. Nunc consectetur commodo ante semper suscipit. Quisque ullamcorper mauris id arcu placerat pellentesque. Sed sit amet dolor metus.

\section{Model}
\label{sec:model}

\subsection{Text Classification}

We propose a deep neural model to capture linguistics paterns in text. This model is based on simple Convolutional Neural Network models with an embedding layer for word representaions, one convolutional with pooling layer and finaly one dense layer. Figure \ref{cnn} shows the global structure of our architecture. The input is a sequence of words $ w_{1}, w_{2} ... w_{n} $ and the output contains class elements (for text classification). The embedding is built on top of a Word2Vec architecture trained on a Skip-gram model. Our text tokenizer keeps all the words to make sure all linguistic material is detected at the end by the model. This embedding is also modifiable by the model to attain optimal text-classification acuracy. 

The Convolutional layer is based on a two-dimensional convolution, the same as used for picture convolution, but with a fixed width corresponding to the max width (this size is actually equal to the embedding size). With this setting, our usage of the two-dimensional convolution is in reallity the same as a one-dimensional convolution (the default convolutionnal layer for text). The only parameter we adjust here is the height of the filter corresponding to the number of words we want to put in the filter. The goal of this approach is to be able to use the standard picture deconvolution (also called transposed convolutions\footnote{transposed convolutions works by swapping the forward and backward passes of a convolution. It's a transformation that going in the opposite direction of a normal convolution}) methods for our model on text.

For our needs, we limited the convolutional part at one layer only. The performances are good enough, but it's scallable and we can add more layers according to the dataset complexity.

The last layer is a fully connected dense network with a softmax finishing on a output size corresponding to the number of classes we attempt to train. The model is trained by cross-entropy with an Adam optimizer.

\begin{figure}[h]
\begin{center}
\includegraphics[width=8cm]{img/model_classif.png}
\caption{CNN model}
\label{cnn}
\end{center}
\end{figure}

\subsection{Deconvolution}

Since we use same architecture as image detection, making a deconvolutional layer is really straightforward. There are several methods to visualize the deep internal mecanisms of a neural network. One is known as convolutional transposed. Our deconvolutional network use the same embedding and convolution layer as we use for the classification but we replace the finale dense layer by a transposed convolution layer. After we trained the model we setup the weight of each neuron of the deconvolutional network with the learned weights of the classification network. The result is a new network that takes as input a sequence of words and gives as output all the trained filters of the text classification applied on the given sequence. Then the activation score of each word is calculated as shown in Equation \ref{equation} with $x$ is the size of the embedding, and $y$ the number of applied filters : 

\begin{equation}
\mathop{\sum^{x}\sum^{y}}_{i=1  j=1}  a_{ij} = s_{n}
\label{equation}
\end{equation}

\begin{figure}[h]
\begin{center}
\includegraphics[width=8cm]{img/model_deconv.png}
\caption{Deconvolution model}
\label{cnn}
\end{center}
\end{figure}

With this method we are able to show a sort of topology of a sequence of words. All words have an unique activation score related to the others. We will see now that this output of the deconvolution gives us much information on how the network makes its final descision (prediction). There are well known linguistic marks encoded inside the networm, as well as more complex patterns based on co-occurrences and possibly also on grammatical and syntaxic analysis.


\section{Experiments}

\subsection{Z-score Versus Activation-score}

Z-score is one of the most used methods in linguistic statistics. It compares the observed frequency of a word with the frequency expected in the case of a "normal" distribution. This calculation readily gives, for example, the most specific vocabulary of a given author in a contrastive corpus. The highest z-scores are the most specific words in this case. This is a simple but strong method for analyzing features of text. It can also be used to classify word sequences according to the global z-score (sum of the score) in the sequence. The mean accuracy of this method on our data set is around 85\%, which confirms z-score is in fact meaningful on contrastive data. On the other hand, most of the time deep learning attains greater than 90\% accuracy in text classification. This means that the training methods can learn also on their own some of the linguistic specificities useful in distinguishing between classes of text or authors. We've seen in work on images that thi is the role of convolution. It learns an abstraction of the data to make classification easier. The question is: what is the nature of this abstraction on text? We will see now that deep learning detects automatically words with hight z-score but apparently this is not the only linguistic structure detected.

\begin{figure}[h]
\begin{center}
\includegraphics[width=7.5cm]{img/z-score_activations.png}
\caption{Z-score versus Activation-score}
\label{comparision}
\end{center}
\end{figure}

The Figure \ref{comparision} shows us a comparison between z-score and activation-score on a sequence extract form our latin corpora (Livy Book XXIII Chap. 26). Here it's an example of specific word use by Livy\footnote{Titus Livius Patavinus -- (64 or 59 BC - AD 12 or 17) -- was a Roman historian.}. As we can see, when the z-score is the highest there is a sort of activation spike around the word \textit{castra}. However, this is not always the case: for example small words as \textit{que}, \textit{ad} and \textit{et} are also high in z-score but they do not activate the network at the same level. We saw in (reference ****) that deeplearning is more sensitive to long words, but we can see also on Figure \ref{comparision} that words like \textit{tenebat}, \textit{multum} or \textit{propius} are totally uncorrelated. The Pearson\footnote{Pearson correlation coefficient measures the linear relationship between two datasets. It has a value between $+1$ and $-1$, where $1$ is total positive linear correlation, $0$ is no linear correlation, and $-1$ is total negative} correlation coefficient tells us that in this sequence there is no correlation between z-score and activation-score (with a Pearson of 0.38). This example is one of the most correlated examples of our dataset, thus deep learning seems to learn more than a simple z-score.

In order to understand what the real linguistic marks found by deeplearning are (the convolution layer), we did several tests on different languages and our model seems to have the same behavior in all of them. We used a French web-platform called Hyperbase\footnote{Hyperbase is an on-line (\textit{http://hyperbase.unice.fr}) linguistic toolbox, which allows the creation of databases from textual corpus and the performing of analysis and calculations such z-score, cooccurrences, PCA, K-Means distance, ... } to perform all the linguistic statistics tests. 

\subsection{Dataset: English}

The first dataset we used for our experiments is the well known IMDB Movie review corpus for sentiment classification. It consists of 25,000 reviews labeled by positive or negative sentiment with around 230,000 words. With the default methods given by Hyperbase, we can easily show the specific vocabulary of each class (positive/negative), according to the z-score. There are for example the words \textit{too}, \textit{bad}, \textit{no} or \textit{boring} as most indicitive of negative sentiment, and the words \textit{and}, \textit{performance}, \textit{powerful} or \textit{best} for positive. 
Is it enough to detect automatically if a new review is positive or not? Let's see an example excerpted from a review from December 2017 (not in the training set) on the last American blockbuster:

\begin{quote}
\textit{[...] \textcolor{red}{\textbf{i enjoyed three moments}} in the film in total , \textcolor{red}{\textbf{and if i am being honest and}} the person \textcolor{red}{\textbf{next to me fell asleep}} in the middle and started snoring during the slow space chasescenes . \textcolor{red}{\textbf{the story failed to}} draw me in and entertain \textcolor{red}{\textbf{me the way}} [...]} 
\end{quote}%%%verify this citation

In general the z-score is enough to predict the class of this kind of comment. But in this case, deeplearning seems to do better, but why? If we sum all the z-scores (for negative and positive), the positive class obtains a greater score than the negative. The words \textit{film}, \textit{and}, \textit{honest} and \textit{entertain} -- with scores 5.38, 12.23, 4 and 2.4 -- make this example positive. Deep learning has activated different parts of this sequence (as we show in bold/red in the exemple). If we take the sub-sequence \textit{and if i am being honest and}, there are two occurences of \textit{and} but the first one is followed by \textit{if} and Hyperbase give us 0.84 for \textit{and if} as a negative class. This is far from the 12.23 in the positive. And if we go further, we can do a co-occurrence analysis on \textit{and if} on the training set. As we see on Figure \ref{and_if}, one of most specific adjectives\footnote{With Hyperbase we can focus on different part of speech.} associated with \textit{and if} is \textit{honest}. Exactly what we found in our example. 

\begin{figure}[h]
\begin{center}
\includegraphics[width=7.5cm]{img/cooc_english2.png}
\caption{co-occurrences analysis of \textit{and if} showed by Hyperbase}
\label{and_if}
\end{center}
\end{figure}

In addition, we have the same behavior with the verb \textit{fall}. There is \textit{asleep} next to him. \textit{asleep} alone is not really specific of negative review (z-score of 1.13). But with the word \textit{fall}, \textit{asleep} become one of the most specific (see the co-occurrences analysis - Figure \ref{fall}).

\begin{figure}[h]
\begin{center}
\includegraphics[width=7.5cm]{img/cooc_english.png}
\caption{co-occurrences analysis of \textit{fall} showed by Hyperbase}
\label{fall}
\end{center}
\end{figure}

The activation-score here confirms that deep learning seems to focus not only on high z-score but on more complex patterns and maybe detects the lemma or the part of speech linked to each word. While the embedding is modifiable during the learning, it's possible that the final word vectors share this kind of information. We will see now that these observations are still valid for other languages and can even be generalized between different activation spikes.

\subsection{Dataset: French}

The French data set consists of political speeches. It's a corpus of 2.5 millions of words of French Presidents from 1958 (with C. de Gaulle, the first President of the Fifth Republic) to 2018 with the first speeches by Macron. In this corpus we removed Macron's speech from the 31st of December 2017, to use it as a test data set. In this speech, the deeplearning network primarily recognizes E. Macron (the training task was to be able to predict the correct President). To achieve this task the deeplearning network seems to succeed in finding really complex patterns specific to E. Macron. For example in this sequence :

\begin{quote}
\textit{[...] notre pays \textcolor{red}{\textbf{advienne à}} l'école pour nos enfants, au travail pour l' ensemble de \textcolor{red}{\textbf{nos concitoyens}} pour le climat pour le quotidien de chacune et chacun d' entre vous . \textcolor{red}{\textbf{Ces transformations profondes}} ont commencé et se \textcolor{red}{\textbf{poursuivront}} avec la même force le même rythme la même intensité [...]} 
\end{quote}

\begin{figure}[h]
\begin{center}
\includegraphics[width=7.5cm]{img/macron_activation.png}
\caption{Deconvolution on E. Macron speech.}
\label{macron_activation}
\end{center}
\end{figure}

The z-score gives a result statistically closer to De Gaulle than to E. Macron. The error in the statistical attribution can be explained by a Gaullist phraseology and the multiplication of linguistic markers strongly indexed with de Gaulle: for example, de Gaulle had the characteristic of making long and literary sentences articulated around conjunctions of coordination as in \textit{et} (z-score = 28 for de Gaulle, two occurrences in the excerpt). His speech was also more conceptual than average, and this resulted in an over-use of the articles defined \textit{le}, \textit{la}, \textit{l\'}, \textit{les}) very numerous in the excerpt(7 occurrences); especially in the feminine singular (\textit{la république}, \textit{la liberté}, \textit{la nation}, \textit{la guerre}, etc., here we have \textit{la même force}, \textit{la même intensité}.

The best results given by deeplearning themselves can surprise the linguist and match perfectly with what is known about the sociolinguistics of Macron's dynamic kind of speeches.

The most important activation zone of the excerpt concerns the nominal syntagm \textit{transformations profondes}. Taken separately, neither of the phrase's two words are very Macronian from a statistical point of view (\textit{transformations} = 1.9 \textit{profondes} = 2.9). Better: the syntagm itself is not attested in the President's learning corpus (0 occurrence). However, it can be seen that the co-occurrence of \textit{transformation} and \textit{profondes} amounts to 4.81 at Macron: so it is not the occurrence of one word alone, or the other, which is Macronian but the simultaneous appearance of both in the same window. The second and complementary activation zones of the excerpt thus concern the two verbs \textit{advienne} and \textit{poursuivront}. From a semantic point of view, the two verbs perfectly conspire, after the phrase \textit{transformations profondes}, to give the necessary dynamic to a discourse that advocates change. But it is the verb tenses (borne by the morphology of the verbs) that appear to be the determining factor in the analysis. The calculation of the grammatical codes co-occurring with the word \textit{transformations} thus indicates that the verbs in the subjunctive and the verbs in the future (and also the nouns) are the privileged codes for Macron (Figure \ref{macron}). 

\begin{figure}[h]
\begin{center}
\includegraphics[width=7.5cm]{img/macron_cooc.png}
\caption{Main part-of-speech cooccurrences for \textit{transformations} showed by Hyperbase}
\label{macron}
\end{center}
\end{figure}


More precisely the algorithm indicates that, for Macron, when \textit{transformation} is associated with a verb in the subjunctive (here \textit{advienne}), then there is usually a verb in the future co-present (here \textit{poursuivront}). \textit{transformations profondes}, \textit{advienne} to the subjunctive, \textit{poursuivront} to the future: all these elements together form a speech promising action, from the mouth of a young and dynamic President. Finally, the graph indicates that \textit{transformations} is especially associated with nouns in the President's speeches: in an extraordinary concentration, the excerpt lists 11 (\textit{pays}, \textit{école}, \textit{enfants}, \textit{travail}, \textit{concitoyens}, \textit{climat}, \textit{quotidien}, \textit{transformations}, \textit{force}, \textit{rythme}, \textit{intensité}).

\subsection{Dataset: Latin}

The last dataset we used is based on Latin. We assembled a contrastive corpus of 2 million words with 22 principle authors writting in classical Latin. As in the French dataset, the learning task here was to be able to predict each author according to new sequences of words. The next example is a excerpt of chapter 26 of the 23th book of Livy:

\begin{quote}
\textit{[...] tutus tenebat se quoad multum ac diu PAD quattuor milia peditum et quingenti equites in supplementum missi ex PAD sunt . tum refecta tandem spe \textcolor{red}{\textbf{castra propius hostem}} mouit classem que et ipse instrui parari que iubet ad insulas maritimam que oram tutandam . in \textcolor{red}{\textbf{ipso impetu}} mouendarum de [...]} 
\end{quote}

The statistics here identify this sequence with Caesar\footnote{Gaius Julius Caesar, 100 BC - 44 BC, usually called Julius Caesar, was a Roman politician and general and a notable author of Latin prose.} but Livy is not far off. As historians, Caesar and Livy share a number of specific words: for example tool words like \textit{se} (reflexive pronoun) or \textit{que} (a coordinator) and prepositions like \textit{in}, \textit{ad}, \textit{ex}, \textit{of}. There are also names like \textit{equites} (cavalry) or \textit{castra} (fortified camp).

The attribution of the sentence to Caesar can not only rely only on z-score: \textit{que} or \textit{in} or \textit{castra}, with differences thereof equivalent or inferior to Livy. On the other hand, the differences of \textit{se}, \textit{ex}, are greater, as is that of \textit{equites}. Two very Caesarian terms undoubtedly make the difference \textit{iubet} (he orders) and \textit{milia} (thousands).

The greater score of \textit{quattuor} (four), \textit{castra}, \textit{hostem} (the enemy), \textit{impetu} (the assault) in Livy are not enough to switch the attribution to this author.

On the other hand, deeplearning activates several zones appearing at the beginning of sentences and corresponding to coherent syntactic structures (for Livy) -- \textit{Tandem reflexes spe castra propius hostem mouit} (then, hope having finally returned, he moved the camp closer to the camp of the enemy) -- despite the fact that \textit{castra} in \textit{hostem mouit} is attested only by Tacitus\footnote{Publius (or Gaius) Cornelius Tacitus, 56 BC - 120 BC, was a senator and a historian of the Roman Empire.}. 

There are also \textit{in ipso metu} (in fear itself), while \textit{in} followed by \textit{metu} is counted one time with Caesar and one time also with Quinte-Curce\footnote{Quintus Curtius Rufus was a Roman historian, probably of the 1st century, his only known and only surviving work being "Histories of Alexander the Great"}.

More complex structures are possibly also detected by deeplearning: the structure \textit{tum} + participates Ablative Absolute (\textit{tum refecta}) is more characteristic of Livy (z-score 3.3 with 8 occurrences) than of Caesar (z-score 1.7 with 3 occurrences), even if it is even more specific of Tacitus (z-score 4.2 with 10 occurrences).

Finally and more likely, the co-occurrence between \textit{castra}, \textit{hostem} and \textit{impetu} may have played a major role: Figure \ref{latin}

\begin{figure}[h]
\begin{center}
\includegraphics[width=7.5cm]{img/cooc_latin.png}
\caption{Specific co-occurrences between \textit{impetu} and \textit{castra} showed by Hyperbase.}
\label{latin}
\end{center}
\end{figure}

With Livy, \textit{impetu} appears as a co-occurrent with the lemmas \textit{HOSTIS} (z-score 9.42) and \textit{CASTRA} (z-score 6.75), while \textit{HOSTIS} only has a gap of 3.41 in Caesar and that \textit{CASTRA} does not appear in the list of co-occurrents.

For \textit{castra}, the first co-occurent for Livy is \textit{HOSTIS} (z-score 22.72), before \textit{CASTRA} (z-score 10.18), \textit{AD} (z-score 10.85), \textit{IN} (z-score 8.21), \textit{IMPETVS} (z-score 7.35), \textit{QUE} (z-score 5.86) ) while in Caesar, \textit{IMPETVS} does not appear and the scores of all other lemmas are lower except \textit{CASTRA} (z-score 15.15), \textit{HOSTIS} (8),  \textit{AD} (10,35), \textit{IN} (5,17), \textit{QUE} (4.79).

Thus, all is as it should be if the deeplearning network manages to simultaniously account for specificity, phrase structure, and co-occurence networks\ldots




\section{Conclusion}

ADT and deep learning may not be foreign continents to each other \ citep {lebart1997}. This contribution by crossing statistical approach and neural network allowed us to identify key passages and perhaps reasons that could feed our textual treatments. If the observables that presided over the detection of key passages by the ADT (the lexical specificities) are known and tested, the zones of activation of the deep learning seem to raise new linguistic observables. Recall that the linguistic matter and the topology of the passages can not return to chance: the zones of activations make it possible to obtain recognition rates of more than 90 \% on the French political speech and 85 \% on the corpus of the LASLA ; either rates equivalent to or higher than the rates obtained by the statistical calculation of the key passages. It remains to improve the model and to understand all the mathematical and linguistic outcomes. The first improvement that we now propose to implement is the injection of morphosyntactic information into the network in order to test ever more complex linguistic patterns.


%\bibliography{acl2018}
%\bibliographystyle{acl_natbib}

\appendix

\section{Supplemental Material}
\label{sec:supplemental}

In order to make our experiments reproductible, we going to detail here all the hyperparameters used in our architecture. The neural network is written in python with the library Keras (an tensorflow as backend). 

The embedding use a Word2Vec implementation given by the gensim Library. Here we use the SkipGram model with a window size of 10 words and output vectors of 128 values (embedding dimension).

The textual datas are tokenized by a homemade tokensizer (witch work on English, Latin and French). The corpus is splited into 50 length sequence of words (punctuation is keeped) and each word is converted inta an uniq vectore of 128 value.

The first layer of our model takes the text sequence (as word vectors) and apply on it a weight corresponding to our WordToVec values. Those weight are still trainable during the train of the model.

The second layer is the convolution, a Conv2D in Keras with 512 filters of size $3*128$ (filtering three words at time), with a Relu activation method. Then, there is the Maxpooling (MaxPooling2D) 

(The deconvolution model is identical until here. We replace the rest of the classification model (Dense) by a transposed convolution (Conv2DTranspose).)

The last layers of the model are a Dense layers. One hidden layer of 100 neurons with a Relu activation and one final layer of size eaqual to the number of class with a softmax activation.

All experiments in this paper share the same architecture and the same hyperparameters, and are trained with a cross-entropy method (with an Adam optimizer) with 90\% of the dataset for the training data and 10\% for the validation. The all the tests in this paper are done with new data not included in the original dataset.

\end{document}
